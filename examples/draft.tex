\documentclass[12pt,a4paper]{article}
\usepackage{amsmath,amssymb,amsthm}
\usepackage{graphicx}
\usepackage{tikz}
\usepackage{algorithm}
\usepackage{algpseudocode}
\usepackage{mathtools}

\newtheorem{theorem}{Theorem}
\newtheorem{lemma}[theorem]{Lemma}
\newtheorem{proposition}[theorem]{Proposition}
\newtheorem{corollary}[theorem]{Corollary}
\newtheorem{definition}{Definition}
\newtheorem{remark}{Remark}

\title{Notes on Spectral Propertys of Random Graphs}
\author{Dr. John Doe}
\date{\today}

\begin{document}

\maketitle

\section{Introduction}
In these notes, we explore the spectral properties of random graphs, with particular emphasis on the eighenvalue distribution of their adjacency matrices. The study of random graphs was initiated by Erdős and Renyi \cite{erdos1960evolution}, and has since become a central topic in discrete mathematics and theoretical computer scince.

\section{Preliminaries}
\begin{definition}
    An Erdős-Rényi random graph $G(n,p)$ is a graph on $n$ vertices where each posible edge is included with probability $p$ independent of all other edges.
\end{definition}

Let $A$ be the adjacency matrix of a graph $G$. The eigenvalues of $A$ are denoted $\lambda_1 \geq \lambda_2 \geq \cdots \geq \lambda_n$.

\section{Spectral Gap in Random Regular Graphs}
For a $d$-regular graph, it is well-known that $\lambda_1 = d$. The spectral gap, defined as $d - \lambda_2$, plays a crucial role in determining the expansion properties of the graph.

\begin{theorem}[Alon-Bopanna]
    For any $d$-regular graph on $n$ vertices, we have
    \begin{equation}
        \lambda_2 \geq 2\sqrt{d-1} - o(1)
    \end{equation}
    where $o(1)$ tends to zero as the diameter of the graph tends to infinity.
\end{theorem}

However, for random $d$-regular graphs, we can establish a stronger result

\begin{theorem}[Friedman's Theorem]
    For any $\epsilon > 0$, a random $d$-regular graph $G$ satisfies
    \begin{equation}
        \lambda_2(G) \leq 2\sqrt{d-1} + \epsilon
    \end{equation}
    with probability approaching 1 as $n \to \infty$.
\end{theorem}

The proof relies on the trace method and careful analysis of the moments of \dots

\section{Concentration of Eigenvalues}
For the Erdős-Rényi random graph $G(n,p)$ with $p = \frac{d}{n}$ for some constant $d > 0$, the eigenvalues of the adjacency matrix exhibit interesting concentration phenomena.

\begin{proposition}
    Let $A$ be the adjacency matrix of $G(n,p)$ with $p = \frac{d}{n}$. Then, as $n \to \infty$:
    \begin{enumerate}
        \item $\lambda_1 = (1+o(1))np = (1+o(1))d$ almost surely
        \item For $i \ge 2$, $|\lambda_i| \leq (2+o(1))\sqrt{np(1-p)} = (2+o(1))\sqrt{d}$ almost surely
    \end{enumerate}
\end{proposition}

This shows that the bulk of the spectrum is concentrated in a band of width $O(\sqrt{d})$ around the origin, while the largest eigenvalue is separated from this band.



\bibliographystyle{plain}
\begin{thebibliography}{9}
    \bibitem{erdos1960evolution}
    P. Erdős and A. Rényi,
    \textit{On the evolution of random graphs},
    Publ. Math. Inst. Hung. Acad. Sci, 5(1):17--60, 1960.

    \bibitem{friedman2008proof}
    J. Friedman,
    \textit{A proof of Alon's second eigenvalue conjecture and related problems},
    Memoirs of the American Mathematical Society, 195(910), 2008.

    \bibitem{wigner1958distribution}
    E. P. Wigner,
    \textit{On the distribution of the roots of certain symmetric matrices},
    Annals of Mathematics, 67(2):325--327, 1958.

    % [TODO: Add more relevant citations, maybe something on Wigner's semicircle law?]

\end{thebibliography}

\end{document}
